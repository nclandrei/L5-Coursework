\documentclass[11pt,english,twocolumn]{article}
\renewcommand{\familydefault}{\sfdefault}
\usepackage[T1]{fontenc}
\usepackage[utf8]{inputenc}
\usepackage{pslatex}
\usepackage[english]{babel}
\usepackage{blindtext}
\usepackage{setspace}
\usepackage{url}
%Definitions from Simon's mya4.sty
% Set the paper size to A4
\setlength{\paperheight}{297mm}
\setlength{\paperwidth}{210mm}
% Define commands which allow the width and height of the text
% to be specified. Centre the text on the page.
\newcommand{\settextwidth}[1]{
\setlength{\textwidth}{#1}
\setlength{\oddsidemargin}{\paperwidth}
\addtolength{\oddsidemargin}{-\textwidth}
\setlength{\oddsidemargin}{0.5\oddsidemargin}
\addtolength{\oddsidemargin}{-1in}
}
\newcommand{\settextheight}[1]{
\setlength{\textheight}{#1}
\setlength{\headheight}{0mm}
\setlength{\headsep}{0mm}
\setlength{\topmargin}{\paperheight}
\addtolength{\topmargin}{-\textheight}
\setlength{\topmargin}{0.5\topmargin}
\addtolength{\topmargin}{-1in}
}
\addtolength{\topsep}{-3mm}% space between first item and preceding paragraph.
\addtolength{\partopsep}{-3mm}% extra space added to \topsep when environment starts a new paragraph.
\addtolength{\itemsep}{-5mm}% space between successive items.

%End of Simon's mya4.sty
\usepackage{graphicx}%This is necessary and it must go after mya4
\settextwidth{176mm}
\settextheight{257mm}
\usepackage[usenames,dvipsnames,svgnames,table]{xcolor}
\def\baselinestretch{0.95}

\usepackage[compact]{titlesec}
\titlespacing{\section}{0pt}{*1}{*1}
\titlespacing{\subsection}{0pt}{*1}{*0}
\titlespacing{\subsubsection}{0pt}{*0}{*0}
\titlespacing{\paragraph}{0pt}{*0}{*1}
\titleformat*{\paragraph}{\itshape}{}{}{}
%% --------------------------------------------------------------------------------------------------------------------------------
\begin{document}
\title{Measuring Quality in Software Tickets using Statistical Analysis}

\author{Andrei-Mihai Nicolae}
\date{}
\maketitle

%====================================================
\section{Introduction}
\label{sec:Introduction}
%====================================================

Software engineering, compared to many other engineering fields, is more
abstract \cite{brooks1995mythical} - in mechanical engineering, for example,
if one wishes to see what is wrong with the engine, the components can be 
touched and manipulated directly by the human operator. However, when taking the example of
compiling and running a simple program written in a language, such as Go 
\cite{golang}, we go through various layers of abstraction which cannot be directly 
seen by the developer: the written characters inside the file get compiled by the
Go compiler to machine readable code; then, the operating system instructs the 
kernel to run the newly generated machine code, which in turn communicates with 
the hardware components responsible for carrying out the necessary computations.

Therefore, having discussed the complex abstraction usually involved in software development,
it is not a trivial task to plan and create applications that fit the requirements of the
stakeholders. Thus, developers have come up with \emph{issue tracking systems}, applications
which help teams plan work, create and assign tasks, monitor progress etc. The basic unit of
functionality these systems make use of is called a \emph{ticket}, which can be usually split
into two main categories: bug reports and feature requests. These software tickets can include 
various types of information, such as:
\begin{itemize}
	\item textual description of the issue/proposed work;
	\item stacktraces that describe the error occured while using the application;
	\item attachments which support the bug report's existence;
	\item story points (i.e. difficulty assigned for that specific ticket, such as 
	how many hours might take a developer to fix it).
\end{itemize}

However, a still unresolved problem arises: what are the \emph{key 
elements} that need to be included when creating a ticket? What makes for a good quality ticket
and how can we measure that quality? Does including a stacktrace help developers fix a bug 
faster rathen than including steps to reproduce the bug?

There are research papers that looked into how to measure quality in software tickets. However,
all of them have taken a qualitative approach through interviews and questionnaires with
developers. I believe that this approach does not fully address the problem and it does not provide
conclusive results, mainly because in this particular scenario, following a quantitative approach
rather than a qualitative one produces both larger volumes of data as well as unbiased results.
Thus, my work is focused on finding the key elements that create quality
in tickets through \emph{statistical analysis} on multiple large-scale open source projects.

%-------------------------------------------
\subsection*{Report Structure}
\label{sec:label-subsection}
%-------------------------------------------

This literature review presents the most relevant papers in the field. Section 
\ref{sec:data-quality} looks at data quality and metrics in the software development area, 
such as data quality-related methodologies that help organizations develop beneficial 
data workcycles; Section \ref{sec:ticket-quality} presents previous work on the field my 
research focuses on, while Section \ref{sec:measuring-waste} presents findings related 
to measuring waste and cost in software projects, such as miscommunication among team 
members and what impact does that have on the organization.

%====================================================
\section{Data Quality and Metrics}
\label{sec:data-quality}
%====================================================

%====================================================
\section{Ticket Quality}
\label{sec:ticket-quality}
%====================================================

%====================================================
\section{Measuring Cost and Waste in Software Projects}
\label{sec:measuring-waste}
%====================================================

\let\oldbibliography\thebibliography
\renewcommand{\thebibliography}[1]{\oldbibliography{#1}
\setlength{\itemsep}{-3pt}}

\bibliographystyle{abbrv}
%\setstretch{0.8}
{
\scriptsize
\bibliography{report}
}
\end{document}
