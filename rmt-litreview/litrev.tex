\documentclass[a4paper,portrait,12pt]{article}
\usepackage[latin1]{inputenc}
\usepackage{calc}
\usepackage{setspace}
\usepackage{fixltx2e}
\usepackage{graphicx}
\usepackage{multicol}
\usepackage{amsmath}

\usepackage{natbib}
\usepackage{color}
\usepackage{hyperref}
 
\author{Andrei-Mihai Nicolae - 2147392}
 
\title{Measuring Software Maintainability}

\begin{document}

% \setlength{\oddsidemargin}{0.9847in-1in}
% \setlength{\textwidth}{\paperwidth - 0.9847in-0.9847in}
  
\maketitle

\section{Introduction}
  
  Software maintenance is the process of modifying an application which has
  already been delivered in order to correct faults, improve usability,
  improve performance and adjust other such attributes\citep{iee1990ieee}. 
 
  An issue that is related to software maintenance is how 
  can it be measured. As software projects grow, the context that 
  surrounds them has a number of variables that need to be considered in order 
  to construct metrics for measuring or predicting maintenance, such as:
    \begin{itemize}
      \item number of developers;
      \item background and experience of developers;
      \item organizational restrictions (e.g. not being able to use open 
      source projects, but only in-house products);
      \item global physical distribution of the company (e.g. offices in
      different continents, thus different time zones).
    \end{itemize}
  The list can be expanded significantly, thus we can infer the importance 
  of developing metrics that can measure software maintenance.

  Having described what are the most important characteristics of software
  maintenance, we need to discuss why research regarding  be undergone, 
  the major reason being reducing overall cost.
  
  Maintenance is one of the costliest processes that revolve around software 
  projects, requiring as much as 75-80\% of the total programming resources 
  throughout the application's lifecycle \citep{lientz1978characteristics}. 
  Even though it has been documented since the beginning of software 
  development, the impact it has on the total cost still remains a big issue 
  for organizations worldwide. 

  Thus, having discussed the importance of software maintainability and the
  reasons why research around this topic is important to the IT industry,
  we can see the benefits of maintainability measurement, such as:
    \begin{itemize}
      \item better understanding of the current state of the software;
      \item better team organization and discipline due to efficient plans 
      which reduce the future fatigue inflicted by heavy maintenance;
      \item knowledge of where fixes need to be applied as soon as possible
      to avoid unnecessary maintenance procedures;
      \item reduced costs for software teams/companies once maintenance 
      culprits are understood and measures taken to avoid unnecessary future
      efforts.
    \end{itemize}

  This research, does, however, have its own limitations. First of all, it does
  not review literature on how software maintenance is defined and what 
  components 
  
  In this literature review we shall inspect three key papers in the field:
    \begin{itemize}
      \item one-page summary of \citet{oman1992metrics} - ;
      \item one-page summary of \citet{pfleeger1990framework} - ;
      \item in-depth review of \citet{li1993maintenance} - the research was one of the
        first that tried to investigate and measure software maintainability efforts. 
        Moreover, it focuses on object-oriented software, which is one of the most
        embraced paradigms in the world nowadays.
    \end{itemize}

\section{The work of \cite{li1993maintenance}}

 \begin{center}
   \begin{tabular}{ | l | r | }
    \hline
    \parbox[t]{5cm}{Title: Maintenance Metrics for the Object Oriented
                    \\ Paradigm} 
     & Name: Andrei-Mihai Nicolae \\ 
     \hline
     Author[s]: Wei Li, Sallie Henry & Matric number: 2147392 \\
    \hline
   \end{tabular}
 \end{center}

\subsection{Abstract}

The object oriented paradigm is one of the most popular approaches to developers
software in the modern world. However, most studies that analyzed software 
metrics in order to assess both the software development process and the
quality of the application focused solely on the procedural paradigm. 
The research presented here tried to inspect what metrics make for a easily
maintainable software project developed using the object oriented paradigm.

\subsection{Problem Statement}

Software maintenance consumes large amounts of company resources and it is
one of the costliest processes that revolve around software development.
As software systems started to emerge worldwide, the first paradigm that
was embraced was the procedural one. However, after a certain time, the need
of object oriented programming for various contexts arose, thus a method of
analyzing and measuring maintenance efforts was needed for such projects.

The approach proposed by the authors tried to solve this issue and, after 
implementing their approach and analysing the data from 2 commercial systems,
they reached the conclusion that the metrics proposed are good predictors
for maintenance effort.

\subsection{Contributions}

The paper presented brings several contributions to the software maintenance
research area. 

The first one is that it takes metrics developed by \citet{chidamber1994metrics} 
and analyses their suitability for object-oriented software applications in the 
context of measuring software maintainability. The second one is that it proposes 
several other object-oriented metrics and, lastly and probably most importantly, it 
uses all these metrics and validates them on real-world commercial systems. 
As most previous work was performed on software implemented using
the procedural paradigm, the work of \citet{li1993maintenance} is a novel research
and brings valuable contributions to the field.

\subsection{Approach}

Their approach was to select a number of metrics from the work of 
\citet{chidamber1994metrics} and test them on the software projects in 
order to examine whether they are good predictors for maintenance efforts 
or not. The selected metrics are:

 \begin{itemize}
  \item Depth of Inheritance Tree - the position of the class in the inheritance tree;
  \item Response for Class - sum of number of local methods and methods calling these
    local methods;
  \item Lack of Cohesion of Class - overall number of disjoint sets composed of 
    local methods;
  \item Weighted Method per Class - summation of the McCabe's cyclomatic complexity
    of all local methods;
  \item Number of Children - number of direct subclasses.
 \end{itemize}

Afterwards, the authors developed their own several other metrics to be used in the 
evaluation process:

 \begin{itemize}
  \item Message Passing Coupling - total number of sends in a class;
  \item Data Abstraction Coupling - number of Abstract Data Types in a class;
  \item Number of Methods - total number of local methods in a class;
  \item SIZE1 - total number of semicolons in a class;
  \item SIZE2 - sum of total number of attributes and total number of local methods 
    in a class.
 \end{itemize}

In the end, they evaluated all metrics on two commercial systems in order to check
whether they are good predictors for software maintainability.

\subsection{Methodology}

They applied their evaluation on software implemented in Classic-Ada 
Design/Programming Language. The maintenance effort data was collected from
2 commercial systems called UIMS (i.e. User Interface Management System)
and QUES (QUality Evaluation System) over a period of three years. The 
maintenance effort is measured by the number of lines changed per class.

The authors then conducted a series of three analyses on the systems in order
to determine two factors: whether the maintenance effort can be predicted from the
metrics and whether the size metrics are the most important predictors.

The first analysis used as a dependent variable \emph{change} and \emph{all metrics}
as independent variables. The question that this analysis wanted to answer was if 
the object oriented metrics under investigation are good predictors. It was designed
as a regression model which measured R-Square values (i.e. regression quality
indicator measuring the quality of predictions) and the p-value (i.e. regression
significance).

The second analysis was again a regression model testing for the same dependent 
variable and the same values (i.e. R-Square and p-value), but the independent 
variables were only the SIZE1 and SIZE2 metrics. This analysis was targeted to 
complement the first analysis in finding the effect of size metrics on overall 
prediction.

The third and final analysis was aimed at finding whether size metrics can be used
in isolation and outperform the full model (i.e. using all metrics). The authors
tested 2 hypotheses: the \emph{null hypothesis} was that there is no difference 
between the size-only model and full model, while the alternative hypothesis was
proposing there is actually a difference between the models. The analysis looked at
Partial F and F critical values for both systems.

\subsection{Outcomes}

After running the evaluation, the authors reached the conclusion that predicting
software maintainability from the proposed metrics is possible. Moreover, the size
metrics (i.e. SIZE1 and SIZE2), even though they are not the main contributors to
overall prediction, they account for a large portion. Finally, another conclusion
the research presents is that the proposed metrics are good predictors.

\subsection{Related Work}
One of the fields related to what the authors are trying to achieve in this
paper is the examination of maintenance metrics in software projects following
the procedural paradigm, some of which are mentioned in the paper:
 \begin{itemize}
  \item \citet{halstead1977elements} software science metrics;
  \item \citet{henry1981software} information flow metrics;
  \item \citet{bail1988program} HAC complexity;
  \item \citet{robillard1989interconnectivity} inter connectivity metric;
  \item \citet{mccabe1976complexity} cyclomatic metrics;
  \item \citet{adamov1990proposal} hybrid metrics.
 \end{itemize}
As the authors were performing a relatively novel research at the time, 
there was little related work similar to what they have examined. However,
the work of \citet{chidamber1994metrics} is recognized as the primary
source of metrics used in the authors' research.

\subsection{Conclusions}
Even though the authors admit the biggest limitation of the study, which is
that they ran evaluations only on two commercial systems, they conducted their research
in a novel topic and reached conclusions that helped following studies. Moreover, 
as object-oriented is one of the most embraced programming paradigms nowadays and 
software maintenance efforts are one of the costliest processes in companies, the
research is a key paper in the field and brings valuable findings, useful even now.
 
\section{The work of \cite{oman1992metrics}}

 \begin{center}
   \begin{tabular}{ | l | r | }
    \hline
    \parbox[t]{5cm}{Title: Metrics for Assessing a Software System's
                    \\  Maintainability} 
     & Name: Andrei-Mihai Nicolae \\ 
     \hline
     Author[s]: Paul Oman, Jack Hagemeister & Matric number: 2147392 \\
    \hline
   \end{tabular}
 \end{center}

\subsection{Summary}

Maintenance is one of the costliest processes that revolve around software projects.
Even though it is easy to define it, it is not so trivial to measure it. However, 
as the artifact we are inspecting is software, there are certain attributes that 
can be analyzed in order to infer an index for overall maintainability. 

The presented research tries to combine these attributes into a hierarchical structure.
Moreover, the authors are also proposing an overall index that can be calculated 
which measures the software's maintainability score.

\subsection{Problem Solved}

The research solves the issue of measuring maintainability in a software. 

\subsection{Approach}

The authors have collected data and findings from over 35 studies and combined
them into form a hierarchical, tree-like structure of attributes. First step was to
define certain categories for the attributes inspected:

\begin{itemize}
  \item software maturity attributes - age, size, stability, maintenance intensity,
    defect intensity, reliability, reuse and subjective product appraisals (e.g.  
    evaluations of programming language complexity, application complexity, 
    development effort expended);
  \item source code - control structure, information structure and typography, 
    naming and commenting;
  \item supporting documentation - documentation abstraction and physical attributes.
\end{itemize}

Then, the tree was defined as \emph{leaf nodes} (i.e. attributes that can be 
actually measured, such as age) and \emph{dimensions} (i.e. groups of leaf nodes).

Finally, the authors' approach has been evaluated through various tools which were
implemented for real-world software systems.

\subsection{Related Work}

As previously mentioned, the study heavily relies on previous metrics presented 
in other studies, coalesced into a previous work of the authors 
\citep{oman1992definition}. 

There are also other studies linked in the paper that try to perform a similar analysis,
among which we can find the work of \citet{basili1983empirical}, 
\citet{peercy1981software}, \citet{selby1989software} or \citet{kafura1987use}.

Furthermore, there are several tools which have used the metrics proposed by the
authors in order to calculate maintainability indexes, such as the work of
\citet{omen1992construction}.

\subsection{Methodology}

After defining the attributes tree-like structure, the authors proposed
a model to unify them and create a universal maintainability score. Then,
they created tools in order to evaluate their metrics on software projects,
such as polynomials for predicting maintainability, a prototype source code
analyzer that calculates maintainability scores for source code written in
C or Pascal and a spreadsheet implementation that inspects attributes which
can not be measured.

\subsection{Conclusions}

This key paper is of great importance to the field of software maintainability
due to its analysis of multiple metrics collected from various previous studies.
Thus, it is not only a proposal for new metrics, but also an analysis of 
previously developed ones and how they fit in the software context. This paper
is still of relevance today due to its comprehensive tree-like structure of 
attributes, thus making it valuable for future studies as well.

\section{The work of \cite{pfleeger1990framework}}

 \begin{center}
   \begin{tabular}{ | l | r | }
    \hline
    \parbox[t]{5cm}{Title: Metrics for Assessing a Software System's
                    \\  Maintainability} 
     & Name: Andrei-Mihai Nicolae \\ 
     \hline
     Author[s]: Paul Oman, Jack Hagemeister & Matric number: 2147392 \\
    \hline
   \end{tabular}
 \end{center}

\subsection{Summary}



\subsection{Problem Solved}

\subsection{Approach}

\subsection{Related Work}

\subsection{Methodology}

\subsection{Conclusions}

\section{Conclusions}

Weigh up the balance of evidence and arguments you have reviewed to say which
positions seem to have the greater empirical and or logical support.

\bibliographystyle{plainnat}
\bibliography{bibliography}

\end{document}
