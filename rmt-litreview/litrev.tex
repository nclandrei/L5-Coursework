\documentclass[a4paper,portrait,12pt]{article}
\usepackage[latin1]{inputenc}
\usepackage{calc}
\usepackage{setspace}
\usepackage{fixltx2e}
\usepackage{graphicx}
\usepackage{multicol}

\usepackage{natbib}
\usepackage{color}
\usepackage{hyperref}
 
\author{Andrei-Mihai Nicolae - 2147392}
 
\title{Measuring Software Maintainability}

\begin{document}

\setlength{\oddsidemargin}{0.9847in-1in}
\setlength{\textwidth}{\paperwidth - 0.9847in-0.9847in}
  
\maketitle

 \section{Introduction}
  
  Software maintenance is the process of modifying an application which has
  already been delivered in order to correct faults, improve usability,
  improve performance and adjust other such attributes\citep{iee1990ieee}. 
 
  An issue that is related to software maintenance is how 
  can it be measured. As software projects grow, the context that 
  surrounds them has a number of variables that need to be considered in order 
  to construct metrics for measuring or predicting maintenance, such as:
    \begin{itemize}
      \item number of developers;
      \item background and experience of developers;
      \item organizational restrictions (e.g. not being able to use open 
      source projects, but only in-house products);
      \item global physical distribution of the company (e.g. offices in
      different continents, thus different time zones).
    \end{itemize}
  The list can be expanded significantly, thus we can infer the importance 
  of developing metrics that can measure software maintenance.

  Having described what are the most important characteristics of software
  maintenance, we need to discuss why research regarding  be undergone, 
  the major reason being reducing overall cost.
  
  Maintenance is one of the costliest processes that revolve around software 
  projects, requiring as much as 75-80\% of the total programming resources 
  throughout the application's lifecycle \citep{lientz1978characteristics}. 
  Even though it has been documented since the beginning of software 
  development, the impact it has on the total cost still remains a big issue 
  for organizations worldwide. 

  Thus, having discussed the importance of software maintainability and the
  reasons why research around this topic is important to the IT industry,
  we can see the benefits of maintainability measurement, such as:
    \begin{itemize}
      \item better understanding of the current state of the software;
      \item better team organization and discipline due to efficient plans 
      which reduce the future fatigue inflicted by heavy maintenance;
      \item knowledge of where fixes need to be applied as soon as possible
      to avoid unnecessary maintenance procedures;
      \item reduced costs for software teams/companies once maintenance 
      culprits are understood and measures taken to avoid unnecessary future
      efforts.
    \end{itemize}
  
  In this literature review we shall review three key papers in the field:
  the work of \citet{oman1992metrics}, the research presented by 
  \citet{pfleeger1990framework}, the work of \citet{li1993maintenance}, the
  research conducted by \citet{yau1985design}.

 \section{The work of Bloggs and Smith}
 
  An in depth presentation of the key ideas and results of the Bloggs/Smith
  school of thought\citep{SYMBOL}. 
 \section{The ideas of Jones}
 
 How the startling results of Jones\citep{Hayes89} completely discredited Smith
 
 \section{The synthesis of  Albert and Zeno}
 ..... As \citet{Einstein} says
 
 and so on until you have covered around half a dozen studies
 \section{Your conclusions}

Weigh up the balance of evidence and arguments you have reviewed to say which
positions seem to have the greater empirical and or logical support.

\bibliographystyle{plainnat}
\bibliography{bibliography}

\end{document}
