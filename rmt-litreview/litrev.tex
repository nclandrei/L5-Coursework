\documentclass[a4paper,portrait,12pt]{article}
\usepackage[latin1]{inputenc}
\usepackage{calc}
\usepackage{setspace}
\usepackage{fixltx2e}
\usepackage{graphicx}
\usepackage{multicol}

\usepackage{natbib}
\usepackage{color}
\usepackage{hyperref}
 
\author{Andrei-Mihai Nicolae - 2147392}
 
\title{Defining, Measuring and Preserving Software Maintainability }

\begin{document}

\setlength{\oddsidemargin}{0.9847in-1in}
\setlength{\textwidth}{\paperwidth - 0.9847in-0.9847in}
  
\maketitle

 \section{Introduction}
  
  Software maintenance is the process of modifying an application which has
  already been delivered in order to correct faults, improve usability,
  improve performance and adjust other such attributes\citep{iee1990ieee}. 
 
  Maintenance is one of the costliest processes that revolve around 
  software projects, requiring as much as 75-80\% of the total programming 
  resources throughout the application's lifecycle
  \citep{lientz1978characteristics}. Even though it has been documented 
  since the beginning of software development, the impact it has on the
  total cost still remains a big issue for organizations worldwide. 

  Moreover, another issue that is related to software maintenance is how 
  can it be measured or predicted. As software projects grow, the context that 
  surrounds them has a number of variables that need to be considered in order 
  to construct metrics for measuring or predicting maintenance, such as:
    \begin{itemize}
      \item number of developers;
      \item background and experience of developers;
      \item organizational restrictions (e.g. not being able to use open 
      source projects, but only in-house products);
      \item global physical distribution of the company (e.g. offices in
      different continents, thus different time zones).
    \end{itemize}
  The list can be expanded significantly, thus we can infer the importance 
  of developing metrics that can measure software maintenance.

  Having described what are the most important characteristics of software
  maintenance, we need to discuss why research regarding this particular
  topic should be undergone. 
 
   \subsection{Main problem 1}
   \subsection{Another important problem}
 \section{The work of Bloggs and Smith}
 
  An in depth presentation of the key ideas and results of the Bloggs/Smith
  school of thought\citep{SYMBOL}. 
 \section{The ideas of Jones}
 
 How the startling results of Jones\citep{Hayes89} completely discredited Smith
 
 \section{The synthesis of  Albert and Zeno}
 ..... As \citet{Einstein} says
 
 and so on until you have covered around half a dozen studies
 \section{Your conclusions}

Weigh up the balance of evidence and arguments you have reviewed to say which
positions seem to have the greater empirical and or logical support.

\bibliographystyle{plainnat}
\bibliography{bibliography}

\end{document}
