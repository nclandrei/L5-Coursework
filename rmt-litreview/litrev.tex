\documentclass[a4paper,portrait,12pt]{article}
\usepackage[latin1]{inputenc}
\usepackage{calc}
\usepackage{setspace}
\usepackage{fixltx2e}
\usepackage{graphicx}
\usepackage{multicol}

\usepackage{natbib}
\usepackage{color}
\usepackage{hyperref}
 
\author{Andrei-Mihai Nicolae - 2147392}
 
\title{Measuring Software Maintainability}

\begin{document}

\setlength{\oddsidemargin}{0.9847in-1in}
\setlength{\textwidth}{\paperwidth - 0.9847in-0.9847in}
  
\maketitle

\section{Introduction}
  
  Software maintenance is the process of modifying an application which has
  already been delivered in order to correct faults, improve usability,
  improve performance and adjust other such attributes\citep{iee1990ieee}. 
 
  An issue that is related to software maintenance is how 
  can it be measured. As software projects grow, the context that 
  surrounds them has a number of variables that need to be considered in order 
  to construct metrics for measuring or predicting maintenance, such as:
    \begin{itemize}
      \item number of developers;
      \item background and experience of developers;
      \item organizational restrictions (e.g. not being able to use open 
      source projects, but only in-house products);
      \item global physical distribution of the company (e.g. offices in
      different continents, thus different time zones).
    \end{itemize}
  The list can be expanded significantly, thus we can infer the importance 
  of developing metrics that can measure software maintenance.

  Having described what are the most important characteristics of software
  maintenance, we need to discuss why research regarding  be undergone, 
  the major reason being reducing overall cost.
  
  Maintenance is one of the costliest processes that revolve around software 
  projects, requiring as much as 75-80\% of the total programming resources 
  throughout the application's lifecycle \citep{lientz1978characteristics}. 
  Even though it has been documented since the beginning of software 
  development, the impact it has on the total cost still remains a big issue 
  for organizations worldwide. 

  Thus, having discussed the importance of software maintainability and the
  reasons why research around this topic is important to the IT industry,
  we can see the benefits of maintainability measurement, such as:
    \begin{itemize}
      \item better understanding of the current state of the software;
      \item better team organization and discipline due to efficient plans 
      which reduce the future fatigue inflicted by heavy maintenance;
      \item knowledge of where fixes need to be applied as soon as possible
      to avoid unnecessary maintenance procedures;
      \item reduced costs for software teams/companies once maintenance 
      culprits are understood and measures taken to avoid unnecessary future
      efforts.
    \end{itemize}

  This research, does, however, have its own limitations. First of all, it does
  not review literature on how software maintenance is defined and what 
  components 
  
  In this literature review we shall inspect three key papers in the field:
    \begin{itemize}
      \item the work of \citet{oman1992metrics} - ;
      \item the research presented by \citet{pfleeger1990framework} - ;
      \item and the work of \citet{li1993maintenance} - ;
    \end{itemize}

\section{The work of \cite{li1993maintenance}}

 \begin{center}
   \begin{tabular}{ | l | r | }
    \hline
    \parbox[t]{5cm}{Title: Maintenance Metrics for the Object Oriented
                    \\ Paradigm} 
     & Name: Andrei-Mihai Nicolae \\ 
     \hline
     Author[s]: Wei Li, Sallie Henry & Matric number: 2147392 \\
    \hline
   \end{tabular}
 \end{center}

\subsection{Summary}
 The object oriented paradigm is one of the most popular approaches to developers
 software in the modern world. However, most studies that analyzed software 
 metrics in order to assess both the software development process and the
 quality of the application focused solely on the procedural paradigm. 
 The research presented here tried to inspect what metrics make for a easily
 maintainable software project developed using the object oriented paradigm.

\subsection{Problem Solved}

\subsection{Approach}

\subsection{Related Work}


\subsection{Methodology}

\subsection{Conclusions}
 
\section{The work of Pfleeger and Bohner}

\section{The work of Pfleeger and Bohner}

\section{Conclusions}

Weigh up the balance of evidence and arguments you have reviewed to say which
positions seem to have the greater empirical and or logical support.

\bibliographystyle{plainnat}
\bibliography{bibliography}

\end{document}
