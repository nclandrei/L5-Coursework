\documentclass[11pt,english,twocolumn]{article}
\renewcommand{\familydefault}{\sfdefault}
\usepackage[T1]{fontenc}
\usepackage[utf8]{inputenc}
\usepackage{pslatex}
\usepackage[english]{babel}
\usepackage{blindtext}
\usepackage{setspace}
\usepackage{url}
%Definitions from Simon's mya4.sty
% Set the paper size to A4
\setlength{\paperheight}{297mm}
\setlength{\paperwidth}{210mm}
% Define commands which allow the width and height of the text
% to be specified. Centre the text on the page.
\newcommand{\settextwidth}[1]{
\setlength{\textwidth}{#1}
\setlength{\oddsidemargin}{\paperwidth}
\addtolength{\oddsidemargin}{-\textwidth}
\setlength{\oddsidemargin}{0.5\oddsidemargin}
\addtolength{\oddsidemargin}{-1in}
}
\newcommand{\settextheight}[1]{
\setlength{\textheight}{#1}
\setlength{\headheight}{0mm}
\setlength{\headsep}{0mm}
\setlength{\topmargin}{\paperheight}
\addtolength{\topmargin}{-\textheight}
\setlength{\topmargin}{0.5\topmargin}
\addtolength{\topmargin}{-1in}
}
\addtolength{\topsep}{-3mm}% space between first item and preceding paragraph.
\addtolength{\partopsep}{-3mm}% extra space added to \topsep when environment starts a new paragraph.
\addtolength{\itemsep}{-5mm}% space between successive items.

%End of Simon's mya4.sty
\usepackage{graphicx}%This is necessary and it must go after mya4
\settextwidth{176mm}
\settextheight{257mm}
\usepackage[usenames,dvipsnames,svgnames,table]{xcolor}
\def\baselinestretch{0.95}

\usepackage[compact]{titlesec}
\titlespacing{\section}{0pt}{*1}{*1}
\titlespacing{\subsection}{0pt}{*1}{*0}
\titlespacing{\subsubsection}{0pt}{*0}{*0}
\titlespacing{\paragraph}{0pt}{*0}{*1}
\titleformat*{\paragraph}{\itshape}{}{}{}
%% --------------------------------------------------------------------------------------------------------------------------------
\begin{document}
\title{Automatic Contextual Bug Report Modelling}

\author{Andrei-Mihai Nicolae (2147392)}
\date{}
\maketitle

%====================================================
\section{Research Problem}

\subsection*{Background}

In recent years, software engineering has been shaping many industries that were previously
considered unrelated to technology, ranging from healthcare to public transportation
and education. As software has grown and inflicted itself into so many fields, it has 
inherently become more complex and it now involves teams of even hundreds of developers working
remotely or in the same office on applications that can shape even the face of a nation, such
as Estonia's data exchange layer called X-Road \cite{x-road}.

Therefore, having such a high number of developers working on complex applications, there 
has been a need for software engineers to plan, track and manage the workload for increased 
efficiency. Thus, issue tracking systems have been created, which usually incorporate multiple
components into a single, unified interface. The main component, however, of any issue tracking system
is called a \emph{ticket}. A ticket is comprised of all the necessary information for either
fixing a bug (i.e. specific malfunction of the application) or implementing a new feature 
(i.e. new functionality for the software). However, after an application is released and 
it starts to get used by actual end users, the most common ticket found in issue tracking systems
is a bug report, stating what went wrong inside the application, how it can be reproduced,
even possible locations in the source code where the malfunction might originate from.

Typically, the people who file the bug reports are actual end users who go onto the project's
issue tracking system website and complete the necessary data required by that system. 
Once the bug report is filed in the tracking systen, triagers are responsible for assigning
that ticket to developers to get fixed. Bettenburg et. al \cite{bettenburg2008makes} have 
conducted interviews with developers and they have gathered valuable data related to what
makes for a good bug report and what are the key characteristics that
help them fix the bug as quickly as possible. One result presented by this research is that
having stack traces (i.e. list of methods in the source code which were called when an exception
was thrown) in the description field of the bug report helps developers fix the bug quicker.
However, even with this information available, the fields displayed in the bug report
for the end users to complete (e.g. summary of the bug, description, attachments) are standard
across all the different issue tracking systems.

Having the optimal bug report model/design for end users to complete would be beneficial for both 
the software development team, as well as the overall costs incurred by the company producing
the application. Therefore, the main goal of this proposed project is to create a tool 
that can automatically generate the optimal bug report model for any specific project taking 
into account the characteristics of that project or the team behind it (e.g. size of the
development team, technologies used, mobile/desktop application). Even though there is 
state-of-the-art research in this field, such as the work of Zimmermann et al.\cite{zimmermann2009improving},
they only look into what can be done and propose only advice to issue tracking systems developers.
However, our work goes one step further and we want to conduct the first research in this area
that actually gives the community a tool which can automatically fix this issue.

\subsection*{Key Ideas in a Nutshell}

\emph{"To automatically create a bug report model tailored specifically
for a project, taking into account the whole context of the software as 
well as the developers behind it."}

The goal of the project is to create a tool which, given as input details
regarding the project and the developers working on it, can automatically
generate a bug report model that will increase the efficiency of the team
as well as reduce the costs of the organization. Our proposed approach is to,
firstly, determine what components determine the time-to-fix period (i.e. time spent
between opening a ticket and closing it) for any given project using both
statistical analysis as well as the data already collected in a publicly 
available appendix \cite{breu2009appendix}. 
Then, in order to make our tool take into account all the complex characteristics
of developing software, we want to both look at how the team of developers is 
managed and how the work is distributed, as well as how the developers interact
with the end users reporting the bugs.

The \emph{main benefit} of our project is that, through fixing a still
unresolved issue in the software engineering field, it will improve three 
key aspects:

\begin{itemize}
	\item reduce the overall costs of the organization developing the software;
	\item improve the time-to-fix window for all bug reports, thus reducing
	maintenance effort;
	\item reduce the communication friction between end users and developers
	because having a tailored bug report model will not require engineers
	constantly wasting time on asking for more details from the users reporting
	the bug.
\end{itemize}

\textbf{Example:} The UK government's Health Digital Services department (as well
as many other departments) is using Jira as an issue tracking system \cite{gov-uk-jira}. 
If our proposed application would be used for their project, the developers 
could streamline the bug reports in a timely manner and increase their productivity,
while also decreasing the overall costs for the government.

\subsection*{Research Objectives}



\subsection*{State of the Art}

\subsection*{Intended Solutions}

\subsection*{Innovative Aspects}

%====================================================
\section{Methodology}

\subsection*{...(WP1)}

\subsection*{...(WP2)}

\subsection*{...(WP3)}

%====================================================
\section{Measurable Outcomes}

%====================================================
\section{Impact and National Importance}

%====================================================
\let\oldbibliography\thebibliography
\renewcommand{\thebibliography}[1]{\oldbibliography{#1}
\setlength{\itemsep}{-3pt}}

\bibliographystyle{abbrv}
%\setstretch{0.8}
{
\scriptsize
\bibliography{proposal}
}
\end{document}
